\documentclass[]{book}
\usepackage{lmodern}
\usepackage{amssymb,amsmath}
\usepackage{ifxetex,ifluatex}
\usepackage{fixltx2e} % provides \textsubscript
\ifnum 0\ifxetex 1\fi\ifluatex 1\fi=0 % if pdftex
  \usepackage[T1]{fontenc}
  \usepackage[utf8]{inputenc}
\else % if luatex or xelatex
  \ifxetex
    \usepackage{mathspec}
  \else
    \usepackage{fontspec}
  \fi
  \defaultfontfeatures{Ligatures=TeX,Scale=MatchLowercase}
\fi
% use upquote if available, for straight quotes in verbatim environments
\IfFileExists{upquote.sty}{\usepackage{upquote}}{}
% use microtype if available
\IfFileExists{microtype.sty}{%
\usepackage{microtype}
\UseMicrotypeSet[protrusion]{basicmath} % disable protrusion for tt fonts
}{}
\usepackage[margin=1in]{geometry}
\usepackage{hyperref}
\hypersetup{unicode=true,
            pdftitle={DISMATH101: Solved Problems in Discrete Math},
            pdfauthor={Melvin Cabatuan},
            pdfborder={0 0 0},
            breaklinks=true}
\urlstyle{same}  % don't use monospace font for urls
\usepackage{natbib}
\bibliographystyle{apalike}
\usepackage{longtable,booktabs}
\usepackage{graphicx,grffile}
\makeatletter
\def\maxwidth{\ifdim\Gin@nat@width>\linewidth\linewidth\else\Gin@nat@width\fi}
\def\maxheight{\ifdim\Gin@nat@height>\textheight\textheight\else\Gin@nat@height\fi}
\makeatother
% Scale images if necessary, so that they will not overflow the page
% margins by default, and it is still possible to overwrite the defaults
% using explicit options in \includegraphics[width, height, ...]{}
\setkeys{Gin}{width=\maxwidth,height=\maxheight,keepaspectratio}
\IfFileExists{parskip.sty}{%
\usepackage{parskip}
}{% else
\setlength{\parindent}{0pt}
\setlength{\parskip}{6pt plus 2pt minus 1pt}
}
\setlength{\emergencystretch}{3em}  % prevent overfull lines
\providecommand{\tightlist}{%
  \setlength{\itemsep}{0pt}\setlength{\parskip}{0pt}}
\setcounter{secnumdepth}{5}
% Redefines (sub)paragraphs to behave more like sections
\ifx\paragraph\undefined\else
\let\oldparagraph\paragraph
\renewcommand{\paragraph}[1]{\oldparagraph{#1}\mbox{}}
\fi
\ifx\subparagraph\undefined\else
\let\oldsubparagraph\subparagraph
\renewcommand{\subparagraph}[1]{\oldsubparagraph{#1}\mbox{}}
\fi

%%% Use protect on footnotes to avoid problems with footnotes in titles
\let\rmarkdownfootnote\footnote%
\def\footnote{\protect\rmarkdownfootnote}

%%% Change title format to be more compact
\usepackage{titling}

% Create subtitle command for use in maketitle
\newcommand{\subtitle}[1]{
  \posttitle{
    \begin{center}\large#1\end{center}
    }
}

\setlength{\droptitle}{-2em}
  \title{DISMATH101: Solved Problems in Discrete Math}
  \pretitle{\vspace{\droptitle}\centering\huge}
  \posttitle{\par}
  \author{Melvin Cabatuan}
  \preauthor{\centering\large\emph}
  \postauthor{\par}
  \predate{\centering\large\emph}
  \postdate{\par}
  \date{2017-05-13}

\usepackage{booktabs}
\usepackage{amssymb,amsmath,amsthm,amsfonts}
\usepackage{mathspec}
\usepackage{tcolorbox}
\newcounter{mycolorbox}[section]
\newenvironment{mycolorbox}{\refstepcounter{mycolorbox}\begin{tcolorbox}[arc=4mm,outer arc=1mm, title=Exercise~\themycolorbox.]}{\end{tcolorbox}}
\newenvironment{solution}{\begin{tcolorbox}[arc=4mm,outer arc=1mm]}{\end{tcolorbox}}
\newtheorem{theorem}{Theorem}
\newtheorem{definition}{Definition}
\makeatletter
\def\thm@space@setup{%
  \thm@preskip=8pt plus 2pt minus 4pt
  \thm@postskip=\thm@preskip
}
\makeatother

\let\BeginKnitrBlock\begin \let\EndKnitrBlock\end
\begin{document}
\maketitle

{
\setcounter{tocdepth}{1}
\tableofcontents
}
\chapter{Logic}\label{logic}

\begin{center}\rule{0.5\linewidth}{\linethickness}\end{center}

\section{Propositional Logic}\label{propositional-logic}

\section{Logical connectives}\label{logical-connectives}

\section{Truth table}\label{truth-table}

\section{Logical Equivalences}\label{logical-equivalences}

\section{Predicate Logic}\label{predicate-logic}

\section{Quantifiers}\label{quantifiers}

\section{Rules of Inference}\label{rules-of-inference}

\begin{center}\rule{0.5\linewidth}{\linethickness}\end{center}

\BeginKnitrBlock{mycolorbox}
The following statements are propositions, EXCEPT\\
A. Euclid of Alexandria is human.\(\hspace{3cm}\:\)C. \(5 > 8\)\\
B. \(1 + 1 = 2\) \(\hspace{6.4cm}\)D. Who am I?
\EndKnitrBlock{mycolorbox}

\begin{quote}
\BeginKnitrBlock{definition}[Proposition]
\protect\hypertarget{def:unnamed-chunk-2}{}{\label{def:unnamed-chunk-2}
\iffalse (Proposition) \fi }a statement that is either true or false
(but not both).
\EndKnitrBlock{definition}
\end{quote}

\begin{quote}
\textbf{Solution:} ``\textbf{D.} Who am I?'' is not a proposition
because interrogative statements have no truth value.
\end{quote}

\BeginKnitrBlock{mycolorbox}
What is the question?
\EndKnitrBlock{mycolorbox}

\begin{quote}
\textbf{Solution:} ``I thoroughly disapprove of duels. If a man should
challenge me, I would take him kindly and forgivingly by the hand and
lead him to a quiet place and kill him.''
\end{quote}

\chapter{Sets}\label{sets}

\begin{center}\rule{0.5\linewidth}{\linethickness}\end{center}

\section{Cardinality of Sets}\label{cardinality-of-sets}

\section{Set Operations}\label{set-operations}

\section{Set Identities}\label{set-identities}

\section{Set Representation}\label{set-representation}

\section{Venn Diagram}\label{venn-diagram}

\section{Cartesian Product}\label{cartesian-product}

\begin{center}\rule{0.5\linewidth}{\linethickness}\end{center}

\chapter{Proofs}\label{proofs}

\begin{center}\rule{0.5\linewidth}{\linethickness}\end{center}

\section{Direct Proof}\label{direct-proof}

\section{Proof by Contraposition}\label{proof-by-contraposition}

\section{Proof by Contradiction}\label{proof-by-contradiction}

\section{Proof by Equivalence}\label{proof-by-equivalence}

\section{Mathematical Induction}\label{mathematical-induction}

\begin{center}\rule{0.5\linewidth}{\linethickness}\end{center}

\chapter{Functions and Relations}\label{functions-and-relations}

\begin{center}\rule{0.5\linewidth}{\linethickness}\end{center}

\section{Basic Types of Functions}\label{basic-types-of-functions}

\section{Composition of Functions}\label{composition-of-functions}

\section{Graphs of Functions}\label{graphs-of-functions}

\section{Relation Properties}\label{relation-properties}

\section{Closures of Relations}\label{closures-of-relations}

\section{Equivalence Relations}\label{equivalence-relations}

\begin{center}\rule{0.5\linewidth}{\linethickness}\end{center}

\chapter{Arrays and Matrices}\label{arrays-and-matrices}

\begin{center}\rule{0.5\linewidth}{\linethickness}\end{center}

\section{Indexing}\label{indexing}

\section{Sequences and Summation}\label{sequences-and-summation}

\section{Matrix Arithmetic}\label{matrix-arithmetic}

\section{Transpose}\label{transpose}

\section{Powers of Matrices}\label{powers-of-matrices}

\section{Determinants}\label{determinants}

\begin{center}\rule{0.5\linewidth}{\linethickness}\end{center}

\chapter{Combinatorics}\label{combinatorics}

\begin{center}\rule{0.5\linewidth}{\linethickness}\end{center}

\section{Basics of Counting}\label{basics-of-counting}

\section{Pigeonhole Principle}\label{pigeonhole-principle}

\section{Permutation and Combination}\label{permutation-and-combination}

\section{Binomial Theorem}\label{binomial-theorem}

\begin{center}\rule{0.5\linewidth}{\linethickness}\end{center}

\chapter{Graphs}\label{graphs}

\begin{center}\rule{0.5\linewidth}{\linethickness}\end{center}

\section{Graph Representation}\label{graph-representation}

\section{Euler and Hamilton Paths}\label{euler-and-hamilton-paths}

\section{Shortest Path}\label{shortest-path}

\section{Planar Graphs}\label{planar-graphs}

\section{Graph Coloring}\label{graph-coloring}

\begin{center}\rule{0.5\linewidth}{\linethickness}\end{center}

\chapter{Trees}\label{trees}

\begin{center}\rule{0.5\linewidth}{\linethickness}\end{center}

\section{Tree Traversal}\label{tree-traversal}

\section{Spanning Trees}\label{spanning-trees}

\section{Binomial Trees}\label{binomial-trees}

\begin{center}\rule{0.5\linewidth}{\linethickness}\end{center}

\chapter{Algorithms}\label{algorithms}

\begin{center}\rule{0.5\linewidth}{\linethickness}\end{center}

\section{Searching}\label{searching}

\section{Sorting}\label{sorting}

\section{Algorithm Paradigms}\label{algorithm-paradigms}

\section{Complexity of Algorithms}\label{complexity-of-algorithms}

\section{Applications}\label{applications}

\begin{center}\rule{0.5\linewidth}{\linethickness}\end{center}

\chapter{Computation Models}\label{computation-models}

\begin{center}\rule{0.5\linewidth}{\linethickness}\end{center}

\section{Finite State Machines}\label{finite-state-machines}

\section{Turing Machine}\label{turing-machine}

\begin{center}\rule{0.5\linewidth}{\linethickness}\end{center}

\bibliography{packages,book}


\end{document}
